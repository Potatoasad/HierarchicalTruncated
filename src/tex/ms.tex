% Define document class
\documentclass[twocolumn]{aastex631}
\usepackage{showyourwork}

% Begin!
\begin{document}

% Title
\title{The Importance of Truncated Gaussian Mixture models for Null tests in a Hierarchical population analysis}

% Author list
\author{Asad Hussain}
\author{Aaron Zimmerman}
\author{Maximilliano Isi}

% Abstract with filler text
\begin{abstract}
Consider any astrophysically relavent variables which are truncated, e.g. spin $0 \leq \chi \leq 1$.  It is often the case that we want to test the hypothesis that the universe conspires to constrain the variable to lie at the boundary, e.g. $\chi = 0$. This is usually seen with beyond GR models which perform null tests of whether the data support or invalidate the theory that deviations from GR are at $0$. The way this is often done is to "recycle" the distribution
Present methods have to choose between the robustness of the fitting distribution vs. it's ability to reproduce the edges. This produces a bias that prevents null tests using population models over truncated variables. We present a framework to 
\end{abstract}

% Main body with filler text
\section{Introduction}
\label{sec:intro}

Lorem ipsum dolor sit amet, consectetuer adipiscing elit.
Ut purus elit, vestibulum ut, placerat ac, adipiscing vitae, felis.
Curabitur dictum gravida mauris, consectetuer id, vulputate a, magna.
Donec vehicula augue eu neque, morbi tristique senectus et netus et.
Mauris ut leo, cras viverra metus rhoncus sem, nulla et lectus vestibulum.
Phasellus eu tellus sit amet tortor gravida placerat.
Integer sapien est, iaculis in, pretium quis, viverra ac, nunc.
Praesent eget sem vel leo ultrices bibendum.
Aenean faucibus, morbi dolor nulla, malesuada eu, pulvinar at, mollis ac.
Curabitur auctor semper nulla donec varius orci eget risus.
Duis nibh mi, congue eu, accumsan eleifend, sagittis quis, diam.
Duis eget orci sit amet orci dignissim rutrum.

Nam dui ligula, fringilla a, euismod sodales, sollici- tudin vel, wisi.
Morbi auctor lorem non justo, nam lacus libero, pretium at, lobortis vitae.
Donec aliquet, tortor sed accumsan bibendum, erat ligula aliquet magna.
Morbi ac orci et nisl hendrerit mollis, suspendisse ut massa, cras nec ante.
Pellentesque a nulla cum sociis natoque penatibus et magnis dis parturient.
Aliquam tincidunt urna, nulla ullamcorper vestibulum turpis.
Pellentesque cursus luctus mauris \citep{Luger2021}.

\bibliography{bib}

\end{document}
